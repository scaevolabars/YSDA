\begin{prb}[Кострикин 8.2 е]
Исследуйте систему и найдите общее решение в зависимости от параметра:\\

\begin{gather*}
\begin{cases}
\lambda x_1 +  x_2 + x_3 &= 1 \\
 x_1 +  \lambda x_2 + x_3 &= 1 \\
 x_1 +  x_2 + \lambda x_3 &= 1 \\
\end{cases}
\end{gather*}

\end{prb}

\begin{sol}
\begin{gather*}
    \left(\begin{array}{c c c|c}
        \lambda & 1 & 1 & 1\\
        1 & \lambda & 1 & 1\\
        1 & 1 & \lambda & 1\\
    \end{array}\right)
    \sim
    \left(\begin{array}{c c c|c}
    1 & \lambda & 1 & 1 \\
    \lambda & 1 & 1 & 1 \\
    1 & 1 & \lambda & 1 \\
    \end{array}\right)
    \sim
    \left(
    \begin{array}{ccc|c}
     1 & \lambda & 1 & 1 \\
     0 & 1  - \lambda^2 & 1 - \lambda & 1 - \lambda \\
     0 & 1 - \lambda & \lambda - 1 & 0 \\
    \end{array}
    \right)
\end{gather*}

При $\lambda = 1$ система имеет решение $x_1 = 1 - x_2 -x_3$

\begin{gather*}
    \lambda \neq 1 \\
    \left(
    \begin{array}{ccc|c}
     1 & \lambda & 1 & 1 \\
     0 & 1  - \lambda^2 & 1 - \lambda & 1 - \lambda \\
     0 & 1 - \lambda & \lambda - 1 & 0 \\
    \end{array}
    \right) \sim 
    \left(
    \begin{array}{ccc|c}
     1 & \lambda & 1 & 1 \\
     0 & 1  + \lambda & 1 & 1 \\
     0 & 1 & -1 & 0 \\
    \end{array}
    \right) \sim
     \left(
    \begin{array}{ccc|c}
     1 & \lambda & 1 & 1 \\
     0 & 0 & \lambda + 2 & 1\\\
     0 & 1 & -1 & 0 \\
    \end{array}
    \right)
\end{gather*}

По теореме Кронекера-Каппели при $\lambda = -2$ система не совместна

\begin{gather*}
    \lambda \neq -2, \lambda \neq 1\\
    \left(
    \begin{array}{ccc|c}
     1 & \lambda & 1 & 1 \\
     0 & 1 & -1 & 0 \\
     0 & 0 & 1 & \frac{1}{\lambda + 2}\\
    \end{array}
    \right)
\end{gather*}
$x_1 = x_2 = x_3 = \frac{1}{\lambda + 2}$
\qed
\end{sol}

\begin{prb}
Найдите многочлен третьей степени $f(x)$ такой, что
\begin{gather*}
    f(0) = -1 \quad 
    f(1) = 1 \quad
    f(-1) = -7 \quad
    f'(1) = 6
\end{gather*}
\end{prb}

\begin{sol}
\begin{gather*}
   \left. f(0) = a_4 + a_3 x + a_2 x^2 + a_1 x^3 \right\vert_{x = 0} = a_4 = -1 \\
   f(1) = a_4 + a_3 + a_2 + a_1 = 1 \\
   f(-1) = a_4 - a_3 x + a_2 -  a_1 = -7 \\
   f'(1) =  a_3 + 2 a_2 + 3 a_3  = 6
\end{gather*}

\begin{gather*}
    \left(
    \begin{array}{cccc|c}
     1 & 1 & 1 & 1 & 1 \\
     1 & -1 & 1 & -1 & -7 \\
     0 & 1 & 2 & 3 & 6 \\
     1 & 0 & 0 & 0 & -1 \\
    \end{array}
    \right)
\end{gather*}

Решая систему получаем $\left(a_4, a_3, a_2, a_1\right) = \left(-1, 1, -2, 3\right)$
\qed
\end{sol}

\begin{nb}

\begin{gather*}
    \left(
    \begin{array}{cc}
    1 & n \\
    0 & 1
    \end{array}
    \right)
    \cdot
    \left(
    \begin{array}{cc}
    1 & m \\
    0 & 1
    \end{array}
    \right) =
    \left(
    \begin{array}{cc}
    1 & n + m \\
    0 & 1
    \end{array}
    \right)
\end{gather*}
\end{nb}

\begin{prb}
Докажите, что $\left(E +  x E_{ij}\right)^k = E +  k x E_{ij}$ при $i \neq j$
\end{prb}

\begin{sol}
Докажем по индукции:
База индукции тривиальна

Предположим, что $\left(E +  x E_{ij}\right)^n = E +  n x E_{ij}$

Докажем, что $\left(E +  x E_{ij}\right)^{n + 1} = E +  (n + 1 )x E_{ij}$

\begin{gather*}
    \left(E +  x E_{ij}\right)^{n + 1} = \left(E +  x E_{ij}\right)^{n} \left(E +  x E_{ij}\right) = \left(E +  n x E_{ij}\right) \left(E +  x E_{ij}\right) = E + (n + 1) x E_{ij} + nx E_{ij}E_{ij} 
\end{gather*}

Слагаемое равно нулю если $i \neq j$ \qed

\end{sol}

\begin{prb}
Доказать что $E_{ij}E_{pq} =  \delta_{ip} E_{jq}$
\end{prb}

\begin{sol}

\end{sol}